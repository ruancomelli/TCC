\usepackage{amsthm}
\usepackage{amsmath}
\usepackage{accents} % For underscripts using command \underaccent
\usepackage{amssymb} % Defines mathematical symbols
\usepackage{amsfonts} % Defines mathematical fonts
\usepackage{bm} % Bold mathematical symbols
\usepackage{physics} % Defines derivative symbols
\usepackage{empheq} % Emphasizes equations
\usepackage{xifthen}
\usepackage{thmtools}
\usepackage{xfrac} % Fractions
\usepackage{interval} % Defines intervals
% \usepackage{braket} % Defines command \set

% Define \(\sqrt[a]{b}\)
\usepackage{letltxmacro}
\makeatletter
\let\oldr@@t\r@@t
\def\r@@t#1#2{%
\setbox0=\hbox{$\oldr@@t#1{#2\,}$}\dimen0=\ht0
\advance\dimen0-0.2\ht0
\setbox2=\hbox{\vrule height\ht0 depth -\dimen0}%
{\box0\lower0.4pt\box2}}
\LetLtxMacro{\oldsqrt}{\sqrt}
\renewcommand*{\sqrt}[2][\ ]{\oldsqrt[#1]{#2} }
\makeatother

\usepackage[decimalsymbol=comma]{siunitx} % This allows numbers to be treated separately "This is a range $x_i \in [\num{-1,5},\num{1,5}]$."
\usepackage{icomma} % This package adds comma as an smart separator, as in ``for $x_{i}\in[-1,5, 1,5]$.''

\newcommand{\bigslant}[2]{
    \left.\raisebox{.2em}{\(#1\)}\middle/\raisebox{-.2em}{\(#2\)}\right.
}

\DeclareSIUnit{\rpm}{rpm}
\DeclareSIUnit{\byte}{B}

\newtheorem{theorem}{Teorema}[chapter]
\newtheorem{lema}{Lema}[chapter]
\newtheorem{definition}{Definição}[chapter]
\newtheorem{corolary}{Corolário}[chapter]
\newtheorem*{unnumberedtheorem}{Teorema}

\DeclareMathOperator{\sign}{sgn}