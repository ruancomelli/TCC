% Floats
\usepackage{float}
\usepackage{subcaption} % Allows us to create figures with multiple images, each one having it's own caption
\usepackage{chngcntr}	% Allows us to number floats according to chapters
\counterwithin{figure}{chapter}
\counterwithin{table}{chapter}

\usepackage{array}
\usepackage{tabularx, tabulary}
\usepackage{tikz}

% Enumerate

% \usepackage{enumerate}

% Allow table elements to spread over multiple rows
\usepackage{multirow}

\usepackage{environ}
\NewEnviron{parametersdesc}
	{
		% \let\olditem\item
		\let\olditem\item
		% \renewcommand{\item}[3]{\olditem[##1: ]\(##2\,##3\);}
		\renewcommand{\item}[3]{##1 & \(##2\) & \(##3\) \tabularnewline}
		% \begin{table}[h]
		\begin{tabularx}{\textwidth}{Xcc}
		\hline
		\multicolumn{1}{c}{Parâmetro} & 
		\multicolumn{1}{c}{Valor} & 
		\multicolumn{1}{c}{Unidade} \tabularnewline
		\hline
		\BODY
		\hline
		\end{tabularx}
		% \end{table}
		% \renewcommand{\item}{\olditem}
	}

\newcommand{\normalresultsfigwidth}{0.9\textwidth}
\newcommand{\smallresultsfigwidth}{0.49\textwidth}
\newcommand{\mine}{do Autor}
\newcommand{\source}[1]{\legend{Fonte: #1.}}
\newcommand{\adapted}[1]{Adaptado de #1}
\newcommand{\sourceMe}{\source{\mine}}