\usepackage{nomencl}
\usepackage{siunitx}
\usepackage{xparse}

\newcommand{\emptyTableEntry}{..}
\newcommand{\dummy}{\cdot}

\newcommand{\predicted}[1]{#1_{\text{pr}}}
% \newcommand{\predicted}[1]{#1^{\,\text{pr}}}
\newcommand{\corrected}[1]{#1_{\text{corr}}}
% \newcommand{\corrected}[1]{#1^{\,\text{corr}}}

\newcommand{\DEM}{DEM}
\newcommand{\CPP}[1]{C\nolinebreak[4]\hspace{-.05em}\raisebox{.4ex}{\relsize{-3}{\textbf{++}}}#1}

\newcommand{\numberOfTimesteps}{N}
\newcommand{\instant}[1][]{t_{#1}}
\newcommand{\Dt}[1][]{\Delta \instant[#1]}
% \newcommand{\eqdef}{:=}
\newcommand*{\eqdef}{\mathrel{\vcenter{\baselineskip0.5ex \lineskiplimit0pt
                     \hbox{\scriptsize.}\hbox{\scriptsize.}}}%
                     =}

% Linear Algebra
\newcommand{\approximately}{\cong}
\renewcommand{\real}{\mathbb{R}}
\renewcommand{\vec}[1]{\bm{#1}}
\newcommand{\vectorFromPoints}[2]{\overrightarrow{#1 #2}}
\newcommand{\normalized}[1]{\dfrac{#1}{\norm{#1}}}
\newcommand{\mtx}[1]{\underline{\bm{#1}}}
\newcommand{\innerProduct}[2]{\left\langle #1, #2 \right\rangle}
\newcommand{\tensor}[1]{\hat{\mathbf{#1}}}
\newcommand{\quaternion}[1]{\bm{#1}}
\newcommand{\drvec}[2]{#2^{\bqty{#1}}}
\newcommand{\drvvec}[2]{\drvec{#1}{\vec{#2}}}
\newcommand{\versor}[1]{\hat{\vec{#1}}}
\newcommand{\nullVector}{\vec{0}}

\newcommand{\originPoint}{O}
\newcommand{\planeAxis}[1]{#1}
\newcommand{\xAxis}{\planeAxis{X}}
\newcommand{\yAxis}{\planeAxis{Y}}
\newcommand{\zAxis}{\planeAxis{Z}}
\newcommand{\xUnit}{\mathbf{i}}
\newcommand{\yUnit}{\mathbf{j}}
\newcommand{\zUnit}{\mathbf{k}}
\newcommand{\xComponent}[1]{{#1}_{\text{x}}}
\newcommand{\yComponent}[1]{{#1}_{\text{y}}}
\newcommand{\zComponent}[1]{{#1}_{\text{z}}}
\newcommand{\explicitVector}[3]{\pqty{#1, #2, #3}}
\newcommand{\explicitNumericVector}[3]{\pqty{#1; #2; #3}}
\newcommand{\explicitVectorCoordinates}[1]{\pqty{\xComponent{#1}, \yComponent{#1}, \zComponent{#1}}}
\newcommand{\explicitVectorPrincipalCoordinates}[1]{\pqty{{#1}_{1}, {#1}_{2}, {#1}_{3}}}

\newcommand{\eqOrder}{p}
\newcommand{\maxDerivOrder}{q}
\newcommand{\taylorOrder}{k}

% Indices
\newcommand{\ind}[2]{{#1}_{#2}} % Add an index #2 to argument #1
% \newcommand{\fromTo}[3]{{#1}_{#3 #2}}
\newcommand{\fromTo}[3]{{#1}_{#2 #3}}
\newcommand{\toFrom}[3]{\fromTo{#1}{#3}{#2}}
\newcommand{\betw}[3]{{#1}_{#2 #3}}

% Physical quantities
\newcommand{\positionScalar}{r}
\newcommand{\positionx}{\xComponent{\positionScalar}}
\newcommand{\positiony}{\yComponent{\positionScalar}}
\newcommand{\positionz}{\zComponent{\positionScalar}}
\newcommand{\position}{\vec{\positionScalar}}
\newcommand{\positioni}{\ind{\position}{i}}
\newcommand{\positionj}{\ind{\position}{j}}

\newcommand{\velocityScalar}{\deriv{1}{\positionScalar}}
\newcommand{\velocityx}{\xComponent{\velocityScalar}}
\newcommand{\velocityy}{\yComponent{\velocityScalar}}
\newcommand{\velocityz}{\zComponent{\velocityScalar}}
\newcommand{\velocity}{\deriv{1}{\position}}
\newcommand{\velocityi}{\ind{\velocity}{i}}
\newcommand{\velocityj}{\ind{\velocity}{j}}

\newcommand{\accelerationScalar}{\deriv{2}{\positionScalar}}
\newcommand{\accelerationx}{\xComponent{\accelerationScalar}}
\newcommand{\accelerationy}{\yComponent{\accelerationScalar}}
\newcommand{\accelerationz}{\zComponent{\accelerationScalar}}
\newcommand{\acceleration}{\deriv{2}{\position}}
\newcommand{\accelerationi}{\ind{\acceleration}{i}}
\newcommand{\accelerationj}{\ind{\acceleration}{j}}
\newcommand{\mass}{m}
\newcommand{\massi}{\ind{\mass}{i}}
\newcommand{\massj}{\ind{\mass}{j}}
\newcommand{\massij}{\betw{\mass}{i}{j}}
\newcommand{\forceScalar}{F}
\newcommand{\force}{\vec{\forceScalar}}
\newcommand{\forceScalarij}{\fromTo{\forceScalar}{i}{j}}
\newcommand{\forceScalarji}{\fromTo{\forceScalar}{j}{i}}
\newcommand{\forcei}{\ind{\force}{i}}
\newcommand{\forcej}{\ind{\force}{j}}
\newcommand{\forceij}{\fromTo{\force}{i}{j}}
\newcommand{\forceji}{\fromTo{\force}{j}{i}}
\newcommand{\resultingForce}{\force_{\text{R}}}
\newcommand{\resultingForcei}{\ind{\force}{i}}
\newcommand{\linearMomentum}{\vec{p}}

\newcommand{\external}[1]{#1^{\text{ext}}}
\newcommand{\internal}[1]{#1^{\text{int}}}

\newcommand{\orientationScalar}{\varphi}
\newcommand{\orientation}{\vec{\orientationScalar}}
\newcommand{\orientationi}{\ind{\orientation}{i}}
\newcommand{\orientationj}{\ind{\orientation}{j}}
\newcommand{\angularVelocityScalar}{\omega}
\newcommand{\angularVelocityVersor}{\versor{\angularVelocityScalar}}
\newcommand{\angularVelocity}{\vec{\angularVelocityScalar}}
\newcommand{\angularVelocityi}{\ind{\angularVelocity}{i}}
\newcommand{\angularVelocityj}{\ind{\angularVelocity}{j}}
\newcommand{\angularAcceleration}{\deriv{1}{\angularVelocity}}
\newcommand{\angularAccelerationi}{\ind{\angularAcceleration}{i}}

\newcommand{\orientationRotationMatrix}{\mtx{\orientationScalar}}

% \newcommand{\parametrizationScalar}{\Phi}
% \newcommand{\parametrization}{\vec{\parametrizationScalar}}
% \newcommand{\parametrizationi}{\ind{\parametrizationScalar}{i}}
% \newcommand{\parametrizationj}{\ind{\parametrizationScalar}{j}}

\newcommand{\momentOfInertia}{J} % Scalar moment of inertia
\newcommand{\momentOfInertiai}{\ind{\momentOfInertia}{i}} % Scalar moment of inertia
\newcommand{\tensorOfInertia}{\tensor{\momentOfInertia}} % Tensorial moment of inertia
\newcommand{\matrixOfInertia}{\mtx{\momentOfInertia}} % Matricial moment of inertia
\newcommand{\xxMomentOfInertia}{\momentOfInertia_{\text{xx}}}
\newcommand{\xyMomentOfInertia}{\momentOfInertia_{\text{xy}}}
\newcommand{\xzMomentOfInertia}{\momentOfInertia_{\text{xz}}}
\newcommand{\yxMomentOfInertia}{\momentOfInertia_{\text{yx}}}
\newcommand{\yyMomentOfInertia}{\momentOfInertia_{\text{yy}}}
\newcommand{\yzMomentOfInertia}{\momentOfInertia_{\text{yz}}}
\newcommand{\zxMomentOfInertia}{\momentOfInertia_{\text{zx}}}
\newcommand{\zyMomentOfInertia}{\momentOfInertia_{\text{zy}}}
\newcommand{\zzMomentOfInertia}{\momentOfInertia_{\text{zz}}}

\newcommand{\rotationMatrix}{\mtx{A}}

\newcommand{\torqueScalar}{M}
\newcommand{\torque}{\vec{\torqueScalar}}
\newcommand{\torquei}{\ind{\torque}{i}}
\newcommand{\torquej}{\ind{\torque}{j}}
\newcommand{\torqueij}{\fromTo{\torque}{i}{j}}
\newcommand{\torqueji}{\fromTo{\torque}{j}{i}}
\newcommand{\resultingTorque}{\torque_{\text{R}}}
\newcommand{\resultingTorquei}{\ind{\torque}{i}}
\newcommand{\angularMomentum}{\vec{L}}

\newcommand{\applied}[3]{\underaccent{#2\to#3}{#1}}

\newcommand*{\IsInteger}[3]{%
    \IfStrEq{#1}{ }{%
        #3% is a blank string
    }{%
        \IfInteger{#1}{#2}{#3}%
    }%
}%
\usepackage{xstring}

\newcommand\lagrangeprime[1]{^{%
\ifcase#1 \or\prime\or\prime\prime\or\prime\prime\prime\else\mathrm{\romannumeral #1}\fi}}

\newcommand\deriv[2]{%
\IfInteger{#1}
{
    \ifcase#1 % 0
        #2
    \or % 1
        \dot{#2}
    \or % 2
        \ddot{#2}
    \or % 3
        \dddot{#2}
    \else
        {#2}^{\mathrm{\romannumeral #1}}
	\fi
}{
    {#2}^{\pqty{#1}}
}
}

\newcommand{\gravityScalar}{g}
\newcommand{\gravity}{\vec{\gravityScalar}}

\newcommand{\initial}[1]{#1^{\text{i}}}
\newcommand{\final}[1]{#1^{\text{f}}}
\newcommand{\initialInstant}{\initial{\instant}}
\newcommand{\finalInstant}{\final{\instant}}

\newcommand{\normal}[1]{#1^{\text{n}}}
\newcommand{\tangential}[1]{#1^{\text{t}}}
\newcommand{\coefficientOfRestitution}{\varepsilon}
\newcommand{\normalCoefficientOfRestitution}{\normal{\coefficientOfRestitution}}
\newcommand{\normalCoefficientOfRestitutionij}{\normal{\betw{\coefficientOfRestitution}{i}{j}}}
\newcommand{\tangentialCoefficientOfRestitution}{\tangential{\coefficientOfRestitution}}
\newcommand{\tangentialCoefficientOfRestitutionij}{\tangential{\betw{\coefficientOfRestitution}{i}{j}}}
\newcommand{\normalVersor}{\normal{\versor{e}}}
\newcommand{\normalVersorij}{\normal{\fromTo{\versor{e}}{i}{j}}}
\newcommand{\normalVersorji}{\normal{\fromTo{\versor{e}}{j}{i}}}
\newcommand{\tangentialVersor}{\tangential{\versor{e}}}
\newcommand{\tangentialVersorij}{\tangential{\fromTo{\versor{e}}{i}{j}}}
\newcommand{\tangentialVersorji}{\tangential{\fromTo{\versor{e}}{j}{i}}}

\newcommand{\radialVectorScalar}{\varrho}
\newcommand{\radialVector}{\vec{\radialVectorScalar}}
\newcommand{\radialVectorMtx}{\mtx{\radialVectorScalar}}

\newcommand{\transpose}[1]{{#1}^T}
\newcommand{\identity}{\mtx{I}}

\newcommand{\basis}{\mtx{e}}
\newcommand{\principal}[1]{#1^\star}

% Elements

\newcommand{\elementSet}{\mathcal{E}}
\newcommand{\element}{\mathcal{E}}
\newcommand{\elementj}{\ind{\element}{j}}

\newcommand{\particleSet}{\mathcal{P}}
\newcommand{\particle}{\mathcal{P}}
\newcommand{\particlei}{\ind{\particle}{i}}
\newcommand{\particlej}{\ind{\particle}{j}}
\newcommand{\particlek}{\ind{\particle}{k}}
\newcommand{\numberOfParticles}{N_{\particleSet}}

\newcommand{\boundarySet}{\mathcal{B}}
\newcommand{\universeSet}{\mathcal{U}}
\newcommand{\domain}{\mathcal{D}}

\newcommand{\set}[1]{\left\lbrace #1 \right\rbrace}
\newcommand{\suchThat}{\,\middle|\,}

\newcommand{\neighborhood}{\mathcal{N}}
\newcommand{\neighborhoodi}{\ind{\neighborhood}{i}}
\newcommand{\neighborhoodOf}[1]{\neighborhood\pqty{#1}}
\newcommand{\ineighborhood}[1]{\neighborhood_{#1}}

\newcommand{\dint}{\displaystyle\int}
\newcommand{\dsum}{\displaystyle\sum}

\newcommand{\distance}{d}
\newcommand{\distanceij}{\betw{\distance}{i}{j}}
\newcommand{\overlap}{\xi}
\newcommand{\overlapij}{\betw{\overlap}{i}{j}}
\newcommand{\overlapDerivative}{\deriv{1}{\overlap}}
\newcommand{\overlapDerivativeij}{\betw{\overlapDerivative}{i}{j}}

\newcommand{\radius}{R}
\newcommand{\radiusi}{\ind{\radius}{i}}
\newcommand{\radiusj}{\ind{\radius}{j}}

\newcommand{\planeOrigin}{\position}
\newcommand{\planeOriginj}{\ind{\planeOrigin}{j}}
\newcommand{\planeNormalVersor}{\versor{n}}
\newcommand{\planeNormalVersorj}{\ind{\planeNormalVersor}{j}}

\newcommand{\normalLine}{N}
\newcommand{\normalLineij}{\betw{\normalLine}{i}{j}}
\newcommand{\contactPlane}{\mathcal{T}}
\newcommand{\contactPlaneij}{\betw{\contactPlane}{i}{j}}
\newcommand{\contactPoint}{P}
\newcommand{\contactPointi}{\ind{\contactPoint}{i}}
\newcommand{\contactPointj}{\ind{\contactPoint}{j}}
\newcommand{\contactPointij}{\betw{\contactPoint}{i}{j}}

\newcommand{\contactRadiusij}{\fromTo{\radialVectorScalar}{i}{j}}
\newcommand{\contactRadiusji}{\fromTo{\radialVectorScalar}{j}{i}}
\newcommand{\contactVectorij}{\fromTo{\radialVector}{i}{j}}
\newcommand{\contactVectorji}{\fromTo{\radialVector}{j}{i}}

\newcommand{\relativeVelocityContactPointScalar}{v}
\newcommand{\relativeVelocityContactPoint}{\vec{\relativeVelocityContactPointScalar}}
\newcommand{\relativeVelocityContactPointScalarij}{\fromTo{\relativeVelocityContactPointScalar}{i}{j}}
\newcommand{\relativeVelocityContactPointij}{\fromTo{\relativeVelocityContactPoint}{i}{j}}
\newcommand{\relativeVelocityContactPointScalarji}{\fromTo{\relativeVelocityContactPointScalar}{j}{i}}
\newcommand{\relativeVelocityContactPointji}{\fromTo{\relativeVelocityContactPoint}{j}{i}}
\newcommand{\relativeNormalVelocityScalarij}{\normal{\fromTo{\relativeVelocityContactPointScalar}{i}{j}}}
\newcommand{\relativeNormalVelocityij}{\normal{\fromTo{\relativeVelocityContactPoint}{i}{j}}}
\newcommand{\relativeNormalVelocityScalarji}{\normal{\fromTo{\relativeVelocityContactPointScalar}{j}{i}}}
\newcommand{\relativeNormalVelocityji}{\normal{\fromTo{\relativeVelocityContactPoint}{j}{i}}}
\newcommand{\relativeTangentialVelocityScalarij}{\tangential{\fromTo{\relativeVelocityContactPointScalar}{i}{j}}}
\newcommand{\relativeTangentialVelocityij}{\tangential{\fromTo{\relativeVelocityContactPoint}{i}{j}}}
\newcommand{\relativeTangentialVelocityScalarji}{\tangential{\fromTo{\relativeVelocityContactPointScalar}{j}{i}}}
\newcommand{\relativeTangentialVelocityji}{\tangential{\fromTo{\relativeVelocityContactPoint}{j}{i}}}

\newcommand{\beforeCollision}[1]{#1}
\newcommand{\afterCollision}[1]{\pqty{#1}^{\prime}}

\newcommand{\normalForceij}{\normal{\forceij}}
\newcommand{\normalForceji}{\normal{\forceji}}
\newcommand{\normalForceScalarij}{\normal{\forceScalarij}}
\newcommand{\normalForceScalarji}{\normal{\forceScalarji}}
\newcommand{\normalElasticForceij}{\normal{\forceScalar_{ij,\text{el}}}}
\newcommand{\normalElasticForceji}{\normal{\forceScalar_{ji,\text{el}}}}
\newcommand{\normalDissipativeForceij}{\normal{\forceScalar_{ij,\text{diss}}}}
\newcommand{\normalDissipativeForceji}{\normal{\forceScalar_{ji,\text{diss}}}}
\newcommand{\tangentialForceij}{\tangential{\forceij}}
\newcommand{\tangentialForceji}{\tangential{\forceji}}
\newcommand{\tangentialForceScalarij}{\tangential{\forceScalarij}}
\newcommand{\tangentialForceScalarji}{\tangential{\forceScalarji}}

\newcommand{\elasticModulus}{Y}
\newcommand{\elasticModulusi}{\ind{\elasticModulus}{i}}
\newcommand{\elasticModulusj}{\ind{\elasticModulus}{j}}
\newcommand{\elasticModulusij}{\betw{\elasticModulus}{i}{j}}
\newcommand{\elasticModulusji}{\betw{\elasticModulus}{j}{i}}

\newcommand{\dampingConstant}{\gamma}
\newcommand{\normalDampingConstant}{\normal{\dampingConstant}}
\newcommand{\normalDampingConstanti}{\normal{\ind{\dampingConstant}{i}}}
\newcommand{\normalDampingConstantj}{\normal{\ind{\dampingConstant}{j}}}
\newcommand{\normalDampingConstantij}{\normal{\betw{\dampingConstant}{i}{j}}}
\newcommand{\normalDampingConstantji}{\normal{\betw{\dampingConstant}{j}{i}}}
\newcommand{\tangentialDampingConstant}{\tangential{\dampingConstant}}
\newcommand{\tangentialDampingConstanti}{\tangential{\ind{\dampingConstant}{i}}}
\newcommand{\tangentialDampingConstantj}{\tangential{\ind{\dampingConstant}{j}}}
\newcommand{\tangentialDampingConstantij}{\tangential{\betw{\dampingConstant}{i}{j}}}
\newcommand{\tangentialDampingConstantji}{\tangential{\betw{\dampingConstant}{j}{i}}}

\newcommand{\normalDissipativeConstant}{A}
\newcommand{\normalDissipativeConstanti}{\ind{\normalDissipativeConstant}{i}}
\newcommand{\normalDissipativeConstantj}{\ind{\normalDissipativeConstant}{j}}
\newcommand{\normalDissipativeConstantij}{\betw{\normalDissipativeConstant}{i}{j}}

\newcommand{\effectiveNormalElasticij}{\normal{\betw{k}{i}{j}}}
\newcommand{\effectiveRadiusij}{\betw{\radius}{i}{j}}

\newcommand{\poisson}{\nu}
\newcommand{\poissoni}{\ind{\poisson}{i}}
\newcommand{\poissonj}{\ind{\poisson}{j}}

\newcommand{\friction}{\mu}
\newcommand{\dynamic}[1]{#1}
\newcommand{\dynamicFrictioni}{\dynamic{\ind{\friction}{i}}}
\newcommand{\dynamicFrictionj}{\dynamic{\ind{\friction}{j}}}
\newcommand{\dynamicFrictionij}{\dynamic{\betw{\friction}{i}{j}}}
\newcommand{\dynamicFrictionForceij}{\dynamic{\fromTo{\force}{i}{j}^{\text{f}}}}
\newcommand{\dynamicFrictionForceji}{\dynamic{\fromTo{\force}{j}{i}^{\text{f}}}}

\newcommand{\elongation}{\zeta}
\newcommand{\elongationij}{\fromTo{\elongation}{i}{j}}
\newcommand{\tangentialElasticConstant}{\kappa^{\text{t}}}
\newcommand{\tangentialElasticConstanti}{\ind{\kappa}{i}^{\text{t}}}
\newcommand{\tangentialElasticConstantj}{\ind{\kappa}{j}^{\text{t}}}
\newcommand{\tangentialElasticConstantij}{\betw{\kappa}{i}{j}^{\text{t}}}

\newcommand{\collisionBeginningij}{\betw{t}{i}{j}}
\newcommand{\collisionEndingij}{\betw{t}{i}{j}^\star}

\newcommand{\betaij}{\betw{\beta}{i}{j}}
\newcommand{\omegaij}{\betw{\omega}{i}{j}}
\newcommand{\omegazij}{\principal{\omegaij}}
\newcommand{\Omegaij}{\betw{\Omega}{i}{j}}

\newcommand{\remainderFunction}[3]{R_{#1,#2,#3}}
\newcommand{\rkty}[3]{\remainderFunction{\taylorOrder}{t}{y}}

\newcommand{\error}{\text{Erro}}
\newcommand{\maximumError}{\underset{\text{max}}{\error}}
\newcommand{\errorOf}[1]{\error\pqty{#1}}
\newcommand{\maximumErrorOf}[1]{\maximumError\pqty{#1}}

\newcommand{\analytical}[1]{#1_S}

\newcommand{\inverted}[1]{\dfrac{1}{#1}}

\newcommand{\eqFor}[1]{f^{#1}}
\newcommand{\eqFori}[1]{{f_i}^{#1}}
\newcommand{\eqForExplicit}[1]{f_{\text{expl}}^{#1}}
\newcommand{\eqForImplicit}[1]{f_{\text{impl}}^{#1}}

\newcommand{\extrapolationMatrixSymbol}{\mtx{A}}
% \newcommand{\extrapolationMatrix}[2]{\extrapolationMatrixSymbol_{#1}}
\NewDocumentCommand{\extrapolationMatrix}{m o}{
    \extrapolationMatrixSymbol_{#1}\IfValueT{#2}{\pqty{#2}}
} % Good explanation on how this works: <https://tex.stackexchange.com/questions/287657/learning-to-use-xparse>. See xparse package

\newcommand{\correctorConstantSymbol}{c}
\newcommand{\correctorConstant}[2]{\correctorConstantSymbol_{#2, #1}}
\newcommand{\correctorConstantVectorSymbol}{\vec{\correctorConstantSymbol}}
\newcommand{\correctorConstantVector}[2]{\correctorConstantVectorSymbol_{#2, #1}}

\newcommand{\dorientation}[1]{\deriv{#1}{\orientation}}
\newcommand{\dorientationi}[1]{\ind{\deriv{#1}{\orientation}}{i}}
\newcommand{\dangularVelocity}[1]{\deriv{#1}{\angularVelocity}}
\newcommand{\dangularVelocityi}[1]{\ind{\deriv{#1}{\angularVelocity}}{i}}
\newcommand{\ipr}{i,\text{pr}}
\newcommand{\icorr}{i,\text{corr}}
\newcommand{\positionipr}{\ind{\position}{\ipr}}
\newcommand{\velocityipr}{\ind{\velocity}{\ipr}}
\newcommand{\accelerationipr}{\ind{\acceleration}{\ipr}}
\newcommand{\positionicorr}{\ind{\position}{\icorr}}
\newcommand{\accelerationicorr}{\ind{\acceleration}{\icorr}}
\newcommand{\orientationipr}{\ind{\orientation}{\ipr}}
\newcommand{\orientationicorr}{\ind{\orientation}{\icorr}}
\newcommand{\dorientationipr}[1]{\ind{\dorientation{#1}}{\ipr}}
\newcommand{\dorientationicorr}[1]{\ind{\dorientation{#1}}{\icorr}}
\newcommand{\jpr}{j,\text{pr}}
\newcommand{\jcorr}{j,\text{corr}}
\newcommand{\positionjpr}{\ind{\position}{\jpr}}
\newcommand{\velocityjpr}{\ind{\velocity}{\jpr}}
\newcommand{\accelerationjpr}{\ind{\acceleration}{\jpr}}
\newcommand{\positionjcorr}{\ind{\position}{\jcorr}}
\newcommand{\orientationjpr}{\ind{\orientation}{\jpr}}
\newcommand{\orientationjcorr}{\ind{\orientation}{\jcorr}}
\newcommand{\dorientationjpr}[1]{\ind{\dorientation{#1}}{\jpr}}
\newcommand{\dorientationjcorr}[1]{\ind{\dorientation{#1}}{\jcorr}}
\newcommand{\angularVelocityipr}{\ind{\angularVelocity}{\ipr}}
\newcommand{\angularVelocityicorr}{\ind{\angularVelocity}{\icorr}}
\newcommand{\angularVelocityjpr}{\ind{\angularVelocity}{\jpr}}
\newcommand{\angularVelocityjcorr}{\ind{\angularVelocity}{\jcorr}}
\newcommand{\dangularVelocityipr}[1]{\ind{\dangularVelocity{#1}}{\ipr}}
\newcommand{\dangularVelocityicorr}[1]{\ind{\dangularVelocity{#1}}{\icorr}}
\newcommand{\drvecipr}[2]{\ind{\drvec{#1}{#2}}{\ipr}}
\newcommand{\drvecjpr}[2]{\ind{\drvec{#1}{#2}}{\jpr}}
\newcommand{\drvecicorr}[2]{\ind{\drvec{#1}{#2}}{\icorr}}
\newcommand{\drvecjcorr}[2]{\ind{\drvec{#1}{#2}}{\jcorr}}

\newcommand{\drag}[1]{#1_D}
\newcommand{\dragForceScalar}{\drag{\forceScalar}}
\newcommand{\dragForce}{\drag{\force}}
\newcommand{\dragCoefficient}{\drag{C}}
\newcommand{\density}{\rho}
\newcommand{\projectedArea}{A}

\newcommand{\verlet}[1]{#1_{\text{V}}}
\newcommand{\verletDistance}{\verlet{\distance}}
\newcommand{\verletPosition}{\verlet{\position}}

\newcommand{\orderedPair}[2]{\pqty{#1, #2}}

% Convention: given a variable \variable, \variablei and \variablej denote \ind{\variable}{i} and \ind{\variable}{j}
% Convention: given a variable \variable whose indices are commutative, \variableij denotes \betw{\variable}{i}{j}
% Convention: given a variable \variable whose indices are not commutative, \variableij denotes \fromTo{\variable}{i}{j}
% Convention: given a variable \variable, \dvariable[1] denotes \deriv{#1}{\variable}