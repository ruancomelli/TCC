\usepackage{nomencl}
\usepackage{siunitx}

\newcommand{\emptyTableEntry}{..}
\newcommand{\dummy}{\cdot}

\newcommand{\predicted}[1]{#1_{\text{pr}}}
% \newcommand{\predicted}[1]{#1^{\,\text{pr}}}
\newcommand{\corrected}[1]{#1_{\text{corr}}}
% \newcommand{\corrected}[1]{#1^{\,\text{corr}}}

\newcommand{\DEM}{DEM}

\newcommand{\Dt}{\Delta t}
% \newcommand{\eqdef}{:=}
\newcommand*{\eqdef}{\mathrel{\vcenter{\baselineskip0.5ex \lineskiplimit0pt
                     \hbox{\scriptsize.}\hbox{\scriptsize.}}}%
                     =}

% Linear Algebra
\newcommand{\approximately}{\cong}
\renewcommand{\real}{\mathbb{R}}
\renewcommand{\vec}[1]{\bm{#1}}
\newcommand{\vectorFromPoints}[2]{\overrightarrow{#1 #2}}
\newcommand{\normalized}[1]{\dfrac{#1}{\norm{#1}}}
\newcommand{\mtx}[1]{\underline{\bm{#1}}}
\newcommand{\innerProduct}[2]{\left\langle #1, #2 \right\rangle}
\newcommand{\tensor}[1]{\hat{\mathbf{#1}}}
\newcommand{\quaternion}[1]{\bm{#1}}
\newcommand{\drvec}[2]{\vec{#2}^{\bqty{#1}}}
\newcommand{\versor}[1]{\hat{\vec{#1}}}
\newcommand{\nullVector}{\vec{0}}

\newcommand{\originPoint}{O}
\newcommand{\planeAxis}[1]{#1}
\newcommand{\xAxis}{\planeAxis{X}}
\newcommand{\yAxis}{\planeAxis{Y}}
\newcommand{\zAxis}{\planeAxis{Z}}
\newcommand{\xUnit}{\mathbf{i}}
\newcommand{\yUnit}{\mathbf{j}}
\newcommand{\zUnit}{\mathbf{k}}
\newcommand{\xComponent}[1]{{#1}_{\text{x}}}
\newcommand{\yComponent}[1]{{#1}_{\text{y}}}
\newcommand{\zComponent}[1]{{#1}_{\text{z}}}
\newcommand{\explicitVector}[3]{\pqty{#1, #2, #3}}
\newcommand{\explicitVectorCoordinates}[1]{\pqty{\xComponent{#1}, \yComponent{#1}, \zComponent{#1}}}
\newcommand{\explicitVectorPrincipalCoordinates}[1]{\pqty{{#1}_{1}, {#1}_{2}, {#1}_{3}}}

\newcommand{\eqOrder}{p}
\newcommand{\maxDerivOrder}{q}
\newcommand{\taylorOrder}{k}

% Indices
\newcommand{\ind}[2]{{#1}_{#2}} % Add an index #2 to argument #1
% \newcommand{\fromTo}[3]{{#1}_{#3 #2}}
\newcommand{\fromTo}[3]{{#1}_{#2 #3}}
\newcommand{\toFrom}[3]{\fromTo{#1}{#3}{#2}}
\newcommand{\betw}[3]{{#1}_{#2 #3}}

% Physical quantities
\newcommand{\positionScalar}{r}
\newcommand{\positionx}{\xComponent{\positionScalar}}
\newcommand{\positiony}{\yComponent{\positionScalar}}
\newcommand{\positionz}{\zComponent{\positionScalar}}
\newcommand{\position}{\vec{\positionScalar}}
\newcommand{\positioni}{\ind{\position}{i}}
\newcommand{\positionj}{\ind{\position}{j}}

\newcommand{\velocityScalar}{\deriv{1}{\positionScalar}}
\newcommand{\velocityx}{\xComponent{\velocityScalar}}
\newcommand{\velocityy}{\yComponent{\velocityScalar}}
\newcommand{\velocityz}{\zComponent{\velocityScalar}}
\newcommand{\velocity}{\deriv{1}{\position}}
\newcommand{\velocityi}{\ind{\velocity}{i}}
\newcommand{\velocityj}{\ind{\velocity}{j}}

\newcommand{\accelerationScalar}{\deriv{2}{\positionScalar}}
\newcommand{\accelerationx}{\xComponent{\accelerationScalar}}
\newcommand{\accelerationy}{\yComponent{\accelerationScalar}}
\newcommand{\accelerationz}{\zComponent{\accelerationScalar}}
\newcommand{\acceleration}{\deriv{2}{\position}}
\newcommand{\mass}{m}
\newcommand{\massi}{\ind{\mass}{i}}
\newcommand{\massj}{\ind{\mass}{j}}
\newcommand{\massij}{\betw{\mass}{i}{j}}
\newcommand{\forceScalar}{F}
\newcommand{\force}{\vec{\forceScalar}}
\newcommand{\forceScalarij}{\fromTo{\forceScalar}{i}{j}}
\newcommand{\forceScalarji}{\fromTo{\forceScalar}{j}{i}}
\newcommand{\forceij}{\fromTo{\force}{i}{j}}
\newcommand{\forceji}{\fromTo{\force}{j}{i}}
\newcommand{\resultingForce}{\force_{\text{R}}}
\newcommand{\resultingForcei}{\ind{\force}{i}}
\newcommand{\linearMomentum}{\vec{p}}

\newcommand{\orientationScalar}{\varphi}
\newcommand{\orientation}{\vec{\orientationScalar}}
\newcommand{\angularVelocityScalar}{\omega}
\newcommand{\angularVelocityVersor}{\versor{\angularVelocityScalar}}
\newcommand{\angularVelocity}{\vec{\angularVelocityScalar}}
\newcommand{\angularVelocityi}{\ind{\angularVelocity}{i}}
\newcommand{\angularVelocityj}{\ind{\angularVelocity}{j}}
\newcommand{\angularAcceleration}{\deriv{1}{\angularVelocity}}

\newcommand{\momentOfInertia}{J} % Scalar moment of inertia
\newcommand{\tensorOfInertia}{\tensor{\momentOfInertia}} % Tensorial moment of inertia
\newcommand{\matrixOfInertia}{\mtx{\momentOfInertia}} % Matricial moment of inertia
\newcommand{\xxMomentOfInertia}{\momentOfInertia_{\text{xx}}}
\newcommand{\xyMomentOfInertia}{\momentOfInertia_{\text{xy}}}
\newcommand{\xzMomentOfInertia}{\momentOfInertia_{\text{xz}}}
\newcommand{\yxMomentOfInertia}{\momentOfInertia_{\text{yx}}}
\newcommand{\yyMomentOfInertia}{\momentOfInertia_{\text{yy}}}
\newcommand{\yzMomentOfInertia}{\momentOfInertia_{\text{yz}}}
\newcommand{\zxMomentOfInertia}{\momentOfInertia_{\text{zx}}}
\newcommand{\zyMomentOfInertia}{\momentOfInertia_{\text{zy}}}
\newcommand{\zzMomentOfInertia}{\momentOfInertia_{\text{zz}}}

\newcommand{\rotationMatrix}{\mtx{A}}

\newcommand{\torqueScalar}{M}
\newcommand{\torque}{\vec{\torqueScalar}}
\newcommand{\torqueij}{\fromTo{\torque}{i}{j}}
\newcommand{\torqueji}{\fromTo{\torque}{j}{i}}
\newcommand{\resultingTorque}{\torque_{\text{R}}}
\newcommand{\resultingTorquei}{\ind{\torque}{i}}
\newcommand{\angularMomentum}{\vec{L}}

\newcommand{\applied}[3]{\underaccent{#2\to#3}{#1}}

\newcommand*{\IsInteger}[3]{%
    \IfStrEq{#1}{ }{%
        #3% is a blank string
    }{%
        \IfInteger{#1}{#2}{#3}%
    }%
}%
\usepackage{xstring}

\newcommand\lagrangeprime[1]{^{%
\ifcase#1 \or\prime\or\prime\prime\or\prime\prime\prime\else\mathrm{\romannumeral #1}\fi}}

\newcommand\deriv[2]{%
\IfInteger{#1}
{
    \ifcase#1 % 0
        #2
    \or % 1
        \dot{#2}
    \or % 2
        \ddot{#2}
    \or % 3
        \dddot{#2}
    \else
        {#2}^{\mathrm{\romannumeral #1}}
	\fi
}{
    {#2}^{\pqty{#1}}
}
}

\newcommand{\gravityScalar}{g}
\newcommand{\gravity}{\vec{\gravityScalar}}

\newcommand{\initial}[1]{#1^{\text{i}}}
\newcommand{\final}[1]{#1^{\text{f}}}

\newcommand{\normal}[1]{#1^{\text{n}}}
\newcommand{\tangential}[1]{#1^{\text{t}}}
\newcommand{\coefficientOfRestitution}{\varepsilon}
\newcommand{\normalCoefficientOfRestitution}{\normal{\coefficientOfRestitution}}
\newcommand{\normalCoefficientOfRestitutionij}{\normal{\betw{\coefficientOfRestitution}{i}{j}}}
\newcommand{\tangentialCoefficientOfRestitution}{\tangential{\coefficientOfRestitution}}
\newcommand{\tangentialCoefficientOfRestitutionij}{\tangential{\betw{\coefficientOfRestitution}{i}{j}}}
\newcommand{\normalVersor}{\normal{\versor{e}}}
\newcommand{\normalVersorij}{\normal{\fromTo{\versor{e}}{i}{j}}}
\newcommand{\normalVersorji}{\normal{\fromTo{\versor{e}}{j}{i}}}
\newcommand{\tangentialVersor}{\tangential{\versor{e}}}
\newcommand{\tangentialVersorij}{\tangential{\fromTo{\versor{e}}{i}{j}}}
\newcommand{\tangentialVersorji}{\tangential{\fromTo{\versor{e}}{j}{i}}}

\newcommand{\radialVectorScalar}{\varrho}
\newcommand{\radialVector}{\vec{\radialVectorScalar}}
\newcommand{\radialVectorMtx}{\mtx{\radialVectorScalar}}

\newcommand{\transpose}[1]{{#1}^T}
\newcommand{\identity}{\mtx{I}}

\newcommand{\basis}{\mtx{e}}
\newcommand{\principal}[1]{{#1}^\star}

\newcommand{\particleSet}{\mathcal{P}}
\newcommand{\particle}{\mathcal{P}}
\newcommand{\particlei}{\ind{\particle}{i}}
\newcommand{\particlej}{\ind{\particle}{j}}
\newcommand{\particlek}{\ind{\particle}{k}}

\newcommand{\element}{\mathcal{E}}
\newcommand{\elementj}{\ind{\element}{j}}
\newcommand{\numberOfParticles}{N_{\particleSet}}

\newcommand{\set}[1]{\left\lbrace #1 \right\rbrace}
\newcommand{\suchThat}{\,\middle|\,}

\newcommand{\neighborhood}{\mathcal{N}}
\newcommand{\neighborhoodOf}[1]{\neighborhood\pqty{#1}}
\newcommand{\ineighborhood}[1]{\neighborhood_{#1}}

\newcommand{\dint}{\displaystyle\int}

\newcommand{\distance}{d}
\newcommand{\distanceij}{\betw{\distance}{i}{j}}
\newcommand{\overlap}{\xi}
\newcommand{\overlapij}{\betw{\overlap}{i}{j}}
\newcommand{\overlapDerivative}{\deriv{1}{\overlap}}
\newcommand{\overlapDerivativeij}{\betw{\overlapDerivative}{i}{j}}

\newcommand{\radius}{R}
\newcommand{\radiusi}{\ind{\radius}{i}}
\newcommand{\radiusj}{\ind{\radius}{j}}

\newcommand{\planeOrigin}{\position}
\newcommand{\planeOriginj}{\ind{\planeOrigin}{j}}
\newcommand{\planeNormalVersor}{\versor{n}}
\newcommand{\planeNormalVersorj}{\ind{\planeNormalVersor}{j}}

\newcommand{\normalLine}{N}
\newcommand{\normalLineij}{\betw{\normalLine}{i}{j}}
\newcommand{\contactPlane}{\mathcal{T}}
\newcommand{\contactPlaneij}{\betw{\contactPlane}{i}{j}}
\newcommand{\contactPoint}{P}
\newcommand{\contactPointi}{\ind{\contactPoint}{i}}
\newcommand{\contactPointj}{\ind{\contactPoint}{j}}
\newcommand{\contactPointij}{\betw{\contactPoint}{i}{j}}
\newcommand{\contactCircleRadius}{a}
\newcommand{\contactCircleRadiusij}{\betw{\contactCircleRadius}{i}{j}}

\newcommand{\contactRadiusij}{\fromTo{\radialVectorScalar}{i}{j}}
\newcommand{\contactRadiusji}{\fromTo{\radialVectorScalar}{j}{i}}
\newcommand{\contactVectorij}{\fromTo{\radialVector}{i}{j}}
\newcommand{\contactVectorji}{\fromTo{\radialVector}{j}{i}}

\newcommand{\relativeVelocityContactPointScalar}{v}
\newcommand{\relativeVelocityContactPoint}{\vec{\relativeVelocityContactPointScalar}}
\newcommand{\relativeVelocityContactPointScalarij}{\fromTo{\relativeVelocityContactPointScalar}{i}{j}}
\newcommand{\relativeVelocityContactPointij}{\fromTo{\relativeVelocityContactPoint}{i}{j}}
\newcommand{\relativeVelocityContactPointScalarji}{\fromTo{\relativeVelocityContactPointScalar}{j}{i}}
\newcommand{\relativeVelocityContactPointji}{\fromTo{\relativeVelocityContactPoint}{j}{i}}
\newcommand{\relativeNormalVelocityScalarij}{\normal{\fromTo{\relativeVelocityContactPointScalar}{i}{j}}}
\newcommand{\relativeNormalVelocityij}{\normal{\fromTo{\relativeVelocityContactPoint}{i}{j}}}
\newcommand{\relativeNormalVelocityScalarji}{\normal{\fromTo{\relativeVelocityContactPointScalar}{j}{i}}}
\newcommand{\relativeNormalVelocityji}{\normal{\fromTo{\relativeVelocityContactPoint}{j}{i}}}
\newcommand{\relativeTangentialVelocityScalarij}{\tangential{\fromTo{\relativeVelocityContactPointScalar}{i}{j}}}
\newcommand{\relativeTangentialVelocityij}{\tangential{\fromTo{\relativeVelocityContactPoint}{i}{j}}}
\newcommand{\relativeTangentialVelocityScalarji}{\tangential{\fromTo{\relativeVelocityContactPointScalar}{j}{i}}}
\newcommand{\relativeTangentialVelocityji}{\tangential{\fromTo{\relativeVelocityContactPoint}{j}{i}}}

\newcommand{\beforeCollision}[1]{#1}
\newcommand{\afterCollision}[1]{\pqty{#1}^{\prime}}

\newcommand{\normalForceij}{\normal{\forceij}}
\newcommand{\normalForceji}{\normal{\forceji}}
\newcommand{\normalForceScalarij}{\normal{\forceScalarij}}
\newcommand{\normalForceScalarji}{\normal{\forceScalarji}}
\newcommand{\normalElasticForceij}{\normal{\forceScalar_{ij,\text{el}}}}
\newcommand{\normalElasticForceji}{\normal{\forceScalar_{ji,\text{el}}}}
\newcommand{\normalDissipativeForceij}{\normal{\forceScalar_{ij,\text{diss}}}}
\newcommand{\normalDissipativeForceji}{\normal{\forceScalar_{ji,\text{diss}}}}
\newcommand{\tangentialForceij}{\tangential{\forceij}}
\newcommand{\tangentialForceji}{\tangential{\forceji}}
\newcommand{\tangentialForceScalarij}{\tangential{\forceScalarij}}
\newcommand{\tangentialForceScalarji}{\tangential{\forceScalarji}}

\newcommand{\elasticModulus}{Y}
\newcommand{\elasticModulusi}{\ind{\elasticModulus}{i}}
\newcommand{\elasticModulusj}{\ind{\elasticModulus}{j}}
\newcommand{\elasticModulusij}{\betw{\elasticModulus}{i}{j}}
\newcommand{\elasticModulusji}{\betw{\elasticModulus}{j}{i}}

\newcommand{\dampingConstant}{\gamma}
\newcommand{\normalDampingConstant}{\normal{\dampingConstant}}
\newcommand{\normalDampingConstanti}{\normal{\ind{\dampingConstant}{i}}}
\newcommand{\normalDampingConstantj}{\normal{\ind{\dampingConstant}{j}}}
\newcommand{\normalDampingConstantij}{\normal{\betw{\dampingConstant}{i}{j}}}
\newcommand{\normalDampingConstantji}{\normal{\betw{\dampingConstant}{j}{i}}}

\newcommand{\normalDissipativeConstant}{A}
\newcommand{\normalDissipativeConstanti}{\ind{\normalDissipativeConstant}{i}}
\newcommand{\normalDissipativeConstantj}{\ind{\normalDissipativeConstant}{j}}
\newcommand{\normalDissipativeConstantij}{\betw{\normalDissipativeConstant}{i}{j}}

\newcommand{\effectiveNormalElasticij}{\normal{\betw{k}{i}{j}}}
\newcommand{\effectiveRadiusij}{\betw{\radius}{i}{j}}

\newcommand{\poisson}{\nu}
\newcommand{\poissoni}{\ind{\poisson}{i}}
\newcommand{\poissonj}{\ind{\poisson}{j}}

% Convention: given a variable \variable, \variablei and \variablej denote \ind{\variable}{i} and \ind{\variable}{j}
% Convention: given a variable \variable whose indices are commutative, \variableij denotes \betw{\variable}{i}{j}
% Convention: given a variable \variable whose indices are not commutative, \variableij denotes \fromTo{\variable}{i}{j}