\chapter{Modelo Matemático}

Segundo \citeonline{bib:maliska}, a solução numérica de qualquer problema físico requer sua prévia modelagem matemática. Um modelo matemático é uma representação de um sistema real através de equações. Essas equações são obtidas ao se fazerem hipóteses sobre o comportamento do sistema estudado, e a representatividade do modelo depende das simplificações feitas nesse processo.

\alert{Falar de métodos explícitos e comentar que o passo de tempo é representado por \(\Dt\)}

\section{Equações de Movimento}

Na dinâmica de partículas, os elementos estudados são considerados corpos rígidos, aos quais aplicam-se as leis de Euler para o movimento \citeay{bib:sampaio}.

Em um sistema inercial fixo, considera-se \particle{} uma partícula de massa \(\mass\) com um centro de massa cuja posição é descrita, em função do tempo, pela função vetorial \(\position\). Além disso, considera-se que a partícula possua um tensor momento de inércia \(\momentOfInertia\) e uma velocidade angular \(\angularVelocity\) definidos com relação à origem do sistema.

Definem-se o \textit{momento linear} e o \textit{momento angular} da partícula, denotados respectivamente por \(\linearMomentum\) e \(\angularMomentum\), como
\begin{gather*}
	\linearMomentum \eqdef \mass\cdot\velocity, \\
	\angularMomentum \eqdef \position\cross\linearMomentum.
\end{gather*}

Ainda, da definição do tensor momento de inércia segue a igualdade
\begin{equation*}
	\angularMomentum = \momentOfInertia\cdot\angularVelocity.
\end{equation*}

\subsection{Primeira Lei de Euler}

A primeira lei de Euler trata do movimento translacional. Ela se expressa através da equação
\begin{equation} \label{eq:euler_first}
	\resultingForce = \deriv{1}{\linearMomentum},
\end{equation}
sendo \(\resultingForce\) o vetor força resultante sobre a partícula. Para o caso em que a massa do corpo é constante, a primeira lei de Euler torna-se equivalente à segunda lei de Newton:
\begin{equation} \label{eq:motion_first}
	\resultingForce = \mass\cdot\acceleration.
\end{equation}

De maneira geral, sistemas de partículas são regidos pela equação \eqref{eq:motion_first}. A equação \eqref{eq:euler_first} é aplicada para os fenômenos de fragmentação e aglutinação, casos em que partículas são divididas ou unidas, sendo que a massa de cada partícula isoladamente não é constante.

\subsection{Segunda Lei de Euler}

A segunda lei de Euler, por sua vez, é o análogo rotacional da primeira:
\begin{equation*} % \label{eq:euler_second}
	\resultingTorque = \deriv{1}{\angularMomentum},
\end{equation*}
em que \(\resultingTorque\) é o torque resultante sobre a partícula calculado com relação à origem do sistema. Para partículas com massa constante, é possível obter
\begin{equation} \label{eq:motion_second}
	\resultingTorque = \deriv{1}{\angularMomentum} = \momentOfInertia\deriv{1}{\angularVelocity} + \angularVelocity\cross\pqty{\momentOfInertia\angularVelocity}.
\end{equation}

Novamente, a hipótese de que a massa da partícula é constante é geralmente levada em conta. 

Entretanto, mesmo diante de situações em que \(m\) é constante, a equação \eqref{eq:motion_second} é problemática
\alert{Falar dos vários sistemas existentes}
\alert{Como abordar a questão dos ângulos de Euler?}

\section{Modelos de Força de Colisão}
\citeonline{bib:sampaio}

\section{Extrapolação de Funções}
\label{sec:extrapolation}

Nos métodos numéricos, a extrapolação de funções possui um papel fundamental por permitir a estimativa dos valores das funções além do conjunto previamente conhecido.

Para simplificação da notação, dada uma função \(y: X\to Y\), define-se
\[\drvec{\taylorOrder}{y} = \pqty{y, \deriv{1}{y}, \deriv{2}{y}, \dots, \deriv{\taylorOrder}{y}}\]
nos pontos em que todas as coordenadas estiverem definidas.

Conforme demonstrado por \citeonline{bib:extrapolation}, métodos de extrapolação lineares para uma função e suas derivadas podem ser escritos na forma
\[
\begin{pmatrix}
	\predicted{y} \\
	\predicted{\deriv{1}{y}} \\
	\predicted{\deriv{2}{y}} \\
	\vdots \\
	\predicted{\deriv{\taylorOrder-1}{y}} \\
	\predicted{\deriv{\taylorOrder}{y}}
\end{pmatrix}
=
\begin{pmatrix}
	a_{0,0} & a_{0,1} & a_{0,2} &  & a_{0,\taylorOrder-1} & a_{0,\taylorOrder} \\
	a_{1,0} & a_{1,1} & a_{1,2} & \cdots & a_{1,\taylorOrder-1} & a_{1,\taylorOrder} \\
	a_{2,0} & a_{2,1} & a_{2,2} &  & a_{2,\taylorOrder-1} & a_{2,\taylorOrder} \\
     & \vdots & & \ddots & & \vdots \\
    a_{\taylorOrder-1,0} & a_{\taylorOrder-1,1} & a_{\taylorOrder-1,2} &  & a_{\taylorOrder-1,\taylorOrder-1} & a_{\taylorOrder-1,\taylorOrder} \\
    a_{\taylorOrder,0} & a_{\taylorOrder,1} & a_{\taylorOrder,2} & \cdots & a_{\taylorOrder,\taylorOrder-1} & a_{\taylorOrder,\taylorOrder}
\end{pmatrix}
\cdot
\begin{pmatrix}
	y \\
	\deriv{1}{y} \\
	\deriv{2}{y} \\
	\vdots \\
	\deriv{\taylorOrder-1}{y} \\
	\deriv{\taylorOrder}{y}
\end{pmatrix}
\]
ou, de forma mais simples,
\begin{equation}
	\drvec{\taylorOrder}{\predicted{y}} = A \cdot \drvec{\taylorOrder}{y}.
\end{equation}
em que a matriz \(A\) é determinada pelo método escolhido e \(\drvec{\taylorOrder}{\predicted{y}}\) é o vetor de derivadas de \(y\) \textit{predito}.

Dentre os métodos extrapolação mais utilizados estão o método de expansão de Taylor, o método de Richardson, o método de interpolação de Aitken e os métodos de Runge-Kutta, cada qual com diferentes características em termos de exatidão e estabilidade \citeay{bib:gear_book}.

\alert{dizer por que escolhemos o de Taylor}

O método de extrapolação por expansão de Taylor é fundamentado pelo \nameref{theo:taylor}.

\begin{theorem}[Teorema  de Taylor] \label{theo:taylor}
	Seja \(y\) uma função com derivadas \(\deriv{1}{y},\dots,y^{\pqty{\taylorOrder+1}}\) todas definidas em um conjunto que contenha \(\bqty{t, t+\Dt}\), e seja \(R_{\taylorOrder, t, y}\) definida por
    \begin{equation*}
    	y(t + \Dt) = y(t) + \deriv{1}{y}(t)\cdot\Dt + \dots + \dfrac{\deriv{\taylorOrder}{y}(t)}{\taylorOrder!}\cdot\Dt^\taylorOrder + R_{\taylorOrder, t, y}(\Dt).
    \end{equation*}
    Então
    \begin{equation} \label{eq:remainder_limit}
    	\lim_{\Dt \rightarrow 0} \dfrac{R_{\taylorOrder, t, y}(\Dt)}{\Dt^\taylorOrder} = 0.
    \end{equation}
\end{theorem}

Uma versão mais completa desse teorema é apresentada e demonstrada por \citeonline{bib:spivak}.

A função \(R_{\taylorOrder, t, y}\) é o resto de ordem \(\taylorOrder\) para a função \(y\) no entorno de \(t\). A equação \eqref{eq:remainder_limit} indica que o resto é um termo da ordem de \(\Dt^{\taylorOrder+1}\), e motiva a aproximação
\begin{equation} \label{eq:taylor_trunc}
    y(t + \Dt) \approximately y(t) + \deriv{1}{y}(t)\cdot\Dt + \dots + \dfrac{\deriv{\taylorOrder}{y}(t)}{\taylorOrder!}\cdot\Dt^\taylorOrder.
\end{equation}

Considerando uma função \(\vec{F}:I\subseteq \real \rightarrow \real^m\), o \nameref{theo:taylor} pode ser aplicado a cada uma de suas funções coordenadas\footnote{Escrevendo \(\vec{F}(t) = \pqty{F_1(t),\dots,F_m(t)}\), a \(i\)-ésima função coordenada de \(\vec{F}\) é a função \(F_i\).}, resultando em uma expansão similar à da equação \eqref{eq:taylor_trunc}. Os casos de interesse são \(m=1\), para funções reais; \(m=2\), para vetores bidimensionais como a posição de uma partícula em uma simulação em duas dimensões; \(m=3\), para simulações em três dimensões; \alert{e \(m=4\) para quaternions}.

Assim, o \nameref{theo:taylor} permite a estimativa do valor de uma função em um ponto \(t+\Dt\) a partir do valor da função e de suas derivadas em um ponto \(t\), e essa estimativa é tanto melhor quanto menor for o valor de \(\Dt\).

Com isso, seja \(\position\) a função posição de uma partícula. Se a posição for conhecida em um instante de tempo \(t\), ela pode ser \textit{prevista} em um instante posterior \(t+\Dt\) explicitamente:
\begin{equation} \label{eq:position_prediction}
	\predicted{\position}\pqty{t+\Dt} = \position\pqty{t} + \Dt\cdot\dv{\position}{t}\pqty{t} + \dfrac{\Dt^2}{2}\cdot \dv[2]{\position}{t}\pqty{t} + \dots + \dfrac{\Dt^\taylorOrder}{\taylorOrder!}\cdot \dv[\taylorOrder]{\position}{t}\pqty{t}.
\end{equation}

Não somente a posição pode ser prevista, mas suas derivadas (como a velocidade e a aceleração) também. Para a \(j\)-ésima derivada de \(\position\):
\[
	\predicted{\deriv{j}{\position}}\pqty{t + \Dt} = \deriv{j}{\position}\pqty{t} + \dots + \dfrac{\Dt^{\taylorOrder-j}}{\pqty{\taylorOrder-j}!}\cdot\deriv{\taylorOrder-j}{\position}\pqty{t}.
\]

Com isso, é possível escrever
\[
	\def\arraystretch{1.2}
\begin{pmatrix}
	\predicted{\position} \\
	\predicted{\deriv{1}{\position}} \\
	\predicted{\deriv{2}{\position}} \\
	\vdots \\
	\predicted{\deriv{\taylorOrder-1}{\position}} \\
	\predicted{\deriv{\taylorOrder}{\position}}
\end{pmatrix}
=
\begin{pmatrix}
	1 & \Dt & \frac{\Dt^2}{2} &  & \frac{\Dt^{\taylorOrder-1}}{\pqty{\taylorOrder-1}!} & \frac{\Dt^\taylorOrder}{\taylorOrder!} \\
	0 & 1 & \Dt & \cdots & \frac{\Dt^{\taylorOrder-2}}{\pqty{\taylorOrder-2}!} & \frac{\Dt^{\taylorOrder-1}}{\pqty{\taylorOrder-1}!} \\
	0 & 0 & 1 &  & \frac{\Dt^{\taylorOrder-3}}{\pqty{\taylorOrder-3}!} & \frac{\Dt^{\taylorOrder-2}}{\pqty{\taylorOrder-2}!} \\
     & \vdots & & \ddots & & \vdots \\
    0 & 0 & 0 &  & 1 & \Dt \\
    0 & 0 & 0 & \cdots & 0 & 1
\end{pmatrix}
\cdot
\begin{pmatrix}
	\position \\
	\deriv{1}{\position} \\
	\deriv{2}{\position} \\
	\vdots \\
	\deriv{\taylorOrder-1}{\position} \\
	\deriv{\taylorOrder}{\position}
\end{pmatrix}.
\]

Esse método de extrapolação ainda pode ser aplicado à função de orientação da partícula e a outros graus de liberdade que o problema porventura exija.

No entanto essa predição geralmente não é exata. Uma das razões para isto é que o truncamento da expansão de Taylor, ou qualquer outro método de extrapolação que se use, despreza a função resto, que não é necessariamente nula. Ainda assim, essa diferença é aceitável quando se utilizam passos de tempo suficientemente pequenos. 

A principal fonte de erros da equação \eqref{eq:position_prediction} é que não se considera, em nenhum momento, a ação de forças externas que porventura atuem sobre a partícula entre os instantes \(t\) e \(t + \Dt\). É necessário, então, \textit{corrigir} a posição prevista. Essa correção pode ser feita através do algoritmo de Gear, apresentado na seção \ref{sec:gear_integration_scheme}.

\section{O Algoritmo de Gear} \label{sec:gear_integration_scheme}

\citeonline{bib:gear_book} considerou o problema de extrapolar funções sujeitas a equações diferenciais. Dada uma função \(y\) tal que \(y(t), \deriv{1}{y}(t),\dots, \deriv{\taylorOrder}{y}(t)\) existem e são bem conhecidos, o objetivo é determinar os valores de \(y(t + \Dt), \deriv{1}{y}(t + \Dt),\dots, \deriv{\taylorOrder}{y}(t + \Dt)\) sabendo que \(y\) deve satisfazer uma equação diferencial da forma
\begin{equation} \label{eq:gear_diff}
	\deriv{\eqOrder}{y} = f\pqty{y, \deriv{1}{y},\dots,\deriv{\eqOrder-1}{y}, t}
\end{equation}
com \(\eqOrder \leq \taylorOrder\).

Do resultado desse estudo originou-se o que é conhecido por algoritmo de Gear \citeay{bib:computational_granular_dynamics}. O algoritmo consiste de duas etapas: a \textit{predição} e a \textit{correção}.

\subsection{Predição}

A etapa de predição é responsável por obter uma estimativa para \(y(t + \Dt)\), \(\deriv{1}{y}(t + \Dt)\),\,\dots, \(\deriv{\taylorOrder}{y}(t + \Dt)\).

A predição é feita através de extrapolações conforme a seção \ref{sec:extrapolation}, obtendo-se uma previsão \(\drvec{\taylorOrder}{\predicted{y}}\) dada por
\[\drvec{\taylorOrder}{\predicted{y}}\pqty{t+\Dt} = A\cdot\drvec{\taylorOrder}{y}\pqty{t}\]

\subsection{Correção}

Na etapa de correção, um termo é adicionado ao vetor previsto \(\drvec{\taylorOrder}{\predicted{y}}\) para se obter o vetor \(\drvec{\taylorOrder}{\corrected{y}}\), corrigido em função da equação diferencial.

O valor corrigido para \(\deriv{\eqOrder}{y}\) pode ser obtido através da equação diferencial \eqref{eq:gear_diff} no instante \(t+\Dt\):
\[
	\corrected{\deriv{\eqOrder}{y}}\pqty{t+\Dt} = f\pqty{\predicted{y}, \predicted{\deriv{1}{y}},\dots,\predicted{\deriv{\eqOrder-1}{y}}, t+\Dt},
\]
e assim é conhecido o valor do erro \(\Delta \deriv{\eqOrder}{y} \eqdef \corrected{\deriv{\eqOrder}{y}} - \predicted{\deriv{\eqOrder}{y}}\) no ponto \(t+\Dt\).

Como demonstrado por \citeonline{bib:gear_book}, as demais derivadas de \(y\) podem ser corrigidas em função de \(\Delta \deriv{\eqOrder}{y}\) conforme a equação
\begin{equation} \label{eq:correction}
	\def\arraystretch{1.2}
	\begin{pmatrix}
		\corrected{\deriv{0}{y}} \\
		\corrected{\deriv{1}{y}} \\
		\corrected{\deriv{2}{y}} \\
		\vdots \\
		\corrected{\deriv{\eqOrder}{y}} \\
		\vdots \\
		\corrected{\deriv{\taylorOrder-1}{y}} \\
		\corrected{\deriv{\taylorOrder}{y}}
	\end{pmatrix}
	=
	\begin{pmatrix}
		\predicted{\deriv{0}{y}} \\
		\predicted{\deriv{1}{y}} \\
		\predicted{\deriv{2}{y}} \\
		\vdots \\
		\predicted{\deriv{\eqOrder}{y}} \\
		\vdots \\
		\predicted{\deriv{\taylorOrder-1}{y}} \\
		\predicted{\deriv{\taylorOrder}{y}}
	\end{pmatrix}
	+
	\begin{pmatrix}
		c_0 \\
		c_1\frac{1}{\Dt} \\
		c_2\frac{2}{\Dt^2} \\
		\vdots \\
		c_p\frac{\eqOrder!}{\Dt^\eqOrder}\\
		\vdots \\
		c_{\taylorOrder-1}\frac{\pqty{\taylorOrder-1}!}{\Dt^{\taylorOrder-1}}\\
		c_{\taylorOrder}\frac{\taylorOrder!}{\Dt^\taylorOrder}
	\end{pmatrix}
	\cdot
	\dfrac{\Dt^\eqOrder}{\eqOrder!}\Delta \deriv{\eqOrder}{y},
\end{equation}
sendo que as constantes corretoras \(c_0\), \(c_1\), \dots, \(c_{\taylorOrder}\) dependem da ordem \(\taylorOrder\) da maior derivada considerada e da ordem \(\eqOrder\) da equação diferencial. Alguns dos valores dessas constantes são apresentados na tabela \ref{table:corrector_constants}.
	
\begin{table}[h]
	\caption{Constantes corretoras para o algoritmo de Gear em função da ordem \(\taylorOrder\) da maior derivada considerada e da ordem \(\eqOrder\) da equação diferencial}
	\label{table:corrector_constants}

	\begin{equation*}
		% \arraycolsep=1.4pt
		\def\arraystretch{1.5}
		\begin{array}{cccccccccc}
	\hline
	\hline
		\eqOrder & \taylorOrder & c_0 & c_1 & c_2 & c_3 & c_4 & c_5 & c_6 & c_7 \\
	\hline
		\multirow{6}{*}{1} 
		& 2 & \frac{5}{12} & 1 & \frac{1}{2} & .. & .. & .. & .. & .. \\
		& 3 & \frac{3}{8} & 1 & \frac{3}{4} & \frac{1}{6} & .. & .. & .. & .. \\
		& 4 & \frac{251}{720} & 1 & \frac{11}{12} & \frac{1}{3} & \frac{1}{24} & .. & .. & .. \\
		& 5 & \frac{95}{288} & 1 & \frac{25}{24} & \frac{35}{72} & \frac{5}{48} & \frac{1}{120} & .. & .. \\
		& 6 & \frac{19087}{60480} & 1 & \frac{137}{120} & \frac{5}{8} & \frac{17}{96} & \frac{1}{40} & \frac{1}{720} & .. \\
		& 7 & \frac{5257}{17280} & 1 & \frac{49}{40} & \frac{203}{270} & \frac{49}{192} & \frac{7}{144} & \frac{7}{1440} & \frac{1}{5040} \\
	\hline
		\multirow{5}{*}{2} 
		& 3 & \frac{1}{6} & \frac{5}{6} & 1 & \frac{1}{3} & .. & .. & .. & .. \\
		& 4 & \frac{19}{120} & \frac{3}{4} & 1 & \frac{1}{2} & \frac{1}{12} & .. & .. & .. \\
		& 5 & \frac{3}{20} & \frac{251}{360} & 1 & \frac{11}{18} & \frac{1}{6} & \frac{1}{60} & .. & .. \\
		& 6 & \frac{863}{6048} & \frac{665}{1008} & 1 & \frac{25}{36} & \frac{35}{144} & \frac{1}{24} & \frac{1}{360} & .. \\
		& 7 & \frac{1925}{14112} & \frac{19087}{30240} & 1 & \frac{137}{180} & \frac{5}{16} & \frac{17}{240} & \frac{1}{120} & \frac{1}{2520} \\
	\hline
		\multirow{4}{*}{3} 
		& 4 & \frac{1}{4} & \frac{1}{2} & \frac{5}{4} & 1 & \frac{1}{4} & .. & .. & .. \\
		& 5 & \frac{3}{80} & \frac{19}{40} & \frac{9}{8} & 1 & \frac{3}{8} & \frac{1}{20} & .. & .. \\
		& 6 & \frac{221}{5040} & \frac{9}{20} & \frac{251}{240} & 1 & \frac{11}{24} & \frac{1}{10} & \frac{1}{120} & .. \\
		& 7 & \frac{2185}{46368} & \frac{863}{2016} & \frac{95}{96} & 1 & \frac{25}{48} & \frac{49}{336} & \frac{1}{48} & \frac{1}{840} \\
	\hline
		\multirow{3}{*}{4} 
		& 5 & \frac{1}{30} & \frac{1}{10} & 1 & \frac{5}{3} & 1 & \frac{1}{5} & .. & .. \\
		& 6 & \frac{16}{630} & \frac{3}{20} & \frac{19}{20} & \frac{3}{2} & 1 & \frac{3}{10} & \frac{1}{30} & .. \\
		& 7 & \frac{11}{630} & \frac{221}{1260} & \frac{9}{10} & \frac{251}{180} & 1 & \frac{11}{30} & \frac{1}{15} & \frac{1}{210} \\
	\hline
	\hline	
		\end{array}
	\end{equation*}
	\legend{Fonte: \citeonline{bib:gear_book}}
\end{table}

\alert{O \(k\) que estou usando é diferente do de Gear. Para Gear, a maior derivada considerada é \(k-1\). A minha tabela está corrigida e, em todos os casos, \(k_{\text{Ruan}} = k_{\text{Gear}}-1 \). Será que devo mudar o símbolo?}

O erro global do algoritmo de Gear é da ordem de \(\Dt^{\taylorOrder+1+\maxDerivOrder-\eqOrder}\), em que \(\maxDerivOrder\) é o maior inteiro tal que a equação diferencial \eqref{eq:gear_diff} pode ser escrita como
\[
	\deriv{\eqOrder}{y} = f\pqty{y, \deriv{1}{y},\dots,\deriv{\eqOrder-\maxDerivOrder}{y}, t}.
\]

Para o caso de uma partícula, \alert{continuar}